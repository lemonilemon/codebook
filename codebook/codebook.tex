<<<<<<< HEAD
\documentclass[a4paper,10pt,twocolumn,oneside]{article}
\setlength{\columnsep}{10pt}                                                                    %兩欄模式的間距
\setlength{\columnseprule}{0pt}                                                                %兩欄模式間格線粗細

\usepackage{amsthm}								%定義,例題
\usepackage{amssymb}
%\usepackage[margin=2cm]{geometry}
\usepackage{fontspec}								%設定字體
\usepackage{color}
\usepackage[x11names]{xcolor}
\usepackage{listings}								%顯示code用的
%\usepackage[Glenn]{fncychap}						%排版,頁面模板
\usepackage{fancyhdr}								%設定頁首頁尾
\usepackage{graphicx}								%Graphic
\usepackage{enumerate}
\usepackage{multicol}
\usepackage{titlesec}
\usepackage{amsmath}
\usepackage[CheckSingle, CJKmath]{xeCJK}
\usepackage{savetrees}
% \usepackage{CJKulem}

%\usepackage[T1]{fontenc}
\usepackage{amsmath, courier, listings, fancyhdr, graphicx}
\topmargin=0pt
\headsep=5pt
\textheight=780pt
\footskip=0pt
\voffset=-40pt
\textwidth=545pt
\marginparsep=0pt
\marginparwidth=0pt
\marginparpush=0pt
\oddsidemargin=0pt
\evensidemargin=0pt
\hoffset=-42pt

\titlespacing\section{0pt}{0pt plus 2pt minus 2pt}{0pt plus 2pt minus 2pt}
\titlespacing\subsection{0pt}{0pt plus 2pt minus 2pt}{0pt plus 2pt minus 2pt}


%\renewcommand\listfigurename{圖目錄}
%\renewcommand\listtablename{表目錄} 

%%%%%%%%%%%%%%%%%%%%%%%%%%%%%

\setmainfont{Ubuntu}				%主要字型
\setmonofont{Ubuntu Mono}
\XeTeXlinebreaklocale "zh"						%中文自動換行
\XeTeXlinebreakskip = 0pt plus 1pt				%設定段落之間的距離
\setcounter{secnumdepth}{3}						%目錄顯示第三層

%%%%%%%%%%%%%%%%%%%%%%%%%%%%%
\newcommand\digitstyle{\color{DarkOrchid3}}
\makeatletter
\lst@CCPutMacro\lst@ProcessOther {"2D}{\lst@ttfamily{-{}}{-{}}}
\@empty\z@\@empty

\newtoks\BBQube@token
\newcount\BBQube@length
\def\BBQube@ResetToken{\BBQube@token{}\BBQube@length\z@}
\def\BBQube@Append#1{\advance\BBQube@length\@ne
  \BBQube@token=\expandafter{\the\BBQube@token#1}}

\def\BBQube@ProcessChar#1{%
  \ifnum\lst@mode=\lst@Pmode%
    \ifnum 9<1#1%
      \expandafter\BBQube@Append{\begingroup\digitstyle #1 \endgroup}%
    \else%
      \expandafter\BBQube@Append{#1}%
    \fi%
  \else%
    \expandafter\BBQube@Append{#1}%
  \fi%
}
\def\BBQube@ProcessStringInner#1#2\BBQube@nil{%
  \expandafter\BBQube@ProcessChar{#1}%
  \if\relax\detokenize{#2}\relax%
  \else%
    \expandafter\BBQube@ProcessStringInner#2\BBQube@nil%
  \fi%
}

\def\BBQube@ProcessString#1{\expandafter\BBQube@ProcessStringInner#1\BBQube@nil}

\lst@AddToHook{OutputOther}{%
\BBQube@ResetToken%
\expandafter\BBQube@ProcessString{\the\lst@token}%
\lst@token=\expandafter{\the\BBQube@token}%
}
\makeatother
\lstset{											% Code顯示
language=C++,										% the language of the code
basicstyle=\footnotesize\ttfamily, 						% the size of the fonts that are used for the code
%numbers=left,										% where to put the line-numbers
numberstyle=\footnotesize,						% the size of the fonts that are used for the line-numbers
stepnumber=1,										% the step between two line-numbers. If it's 1, each line  will be numbered
numbersep=5pt,										% how far the line-numbers are from the code
backgroundcolor=\color{white},					% choose the background color. You must add \usepackage{color}
showspaces=false,									% show spaces adding particular underscores
showstringspaces=false,							% underline spaces within strings
showtabs=false,									% show tabs within strings adding particular underscores
frame=false,											% adds a frame around the code
tabsize=2,											% sets default tabsize to 2 spaces
captionpos=b,										% sets the caption-position to bottom
breaklines=true,									% sets automatic line breaking
breakatwhitespace=false,							% sets if automatic breaks should only happen at whitespace
escapeinside={\%*}{*)},							% if you want to add a comment within your code
morekeywords={constexpr},									% if you want to add more keywords to the set
keywordstyle=\bfseries\color{Blue1},
commentstyle=\itshape\color{Red4},
stringstyle=\itshape\color{Green4},
}

%%%%%%%%%%%%%%%%%%%%%%%%%%%%%

\begin{document}
\pagestyle{fancy}
\fancyfoot{}
%\fancyfoot[R]{\includegraphics[width=20pt]{ironwood.jpg}}
\fancyhead[L]{National Taiwan University 8BQube}
\fancyhead[R]{\thepage}
\renewcommand{\headrulewidth}{0.4pt}
\renewcommand{\contentsname}{Contents} 

\textbf{
\scriptsize
\begin{multicols}{2}
  \tableofcontents
\end{multicols}
}
%%%%%%%%%%%%%%%%%%%%%%%%%%%%%

%\newpage

\footnotesize
\section{Basic}
	\subsection{.vimrc}
	\lstinputlisting[firstline=1]{1_Basic/.vimrc}
	\subsection{PBDS}
	\lstinputlisting[firstline=2]{1_Basic/PBDS.cpp}
	\subsection{pragma}
	\lstinputlisting[firstline=2]{1_Basic/pragma.cpp}
	\subsection{LambdaCompare}
	\lstinputlisting[firstline=2]{1_Basic/lambda_cmp.cpp}
\section{Graph}
	\subsection{2SAT/SCC}
	\lstinputlisting[firstline=2]{2_Graph/2SAT.cpp}
	\subsection{BCC Vertex}
	\lstinputlisting[firstline=2]{2_Graph/BCC_Vertex.cpp}
	\subsection{VirtualTree}
	\lstinputlisting[firstline=2]{2_Graph/Virtual_Tree.cpp}
	\subsection{MinimumMeanCycle}
	\lstinputlisting[firstline=1]{2_Graph/MinimumMeanCycle.cpp}
	\subsection{MaximumCliqueDyn}
	\lstinputlisting[firstline=2]{2_Graph/Maximum_Clique_Dyn.cpp}
	\subsection{DominatorTree}
	\lstinputlisting[firstline=2]{2_Graph/Dominator_Tree.cpp}
	\subsection{DMST(slow)}
	\lstinputlisting[firstline=2]{2_Graph/Minimum_Arborescence.cpp}
	\subsection{DMST}
	\lstinputlisting[firstline=2]{2_Graph/DMST.cpp}
	\subsection{VizingTheorem}
	\lstinputlisting[firstline=2]{2_Graph/Vizing.cpp}
	\subsection{MinimumCliqueCover}
	\lstinputlisting[firstline=3]{2_Graph/Minimum_Clique_Cover.cpp}
	\subsection{CountMaximalClique}
	\lstinputlisting[firstline=2]{2_Graph/NumberofMaximalClique.cpp}
	\subsection{Theorems}
	$|\text{max independent edge set}|=|V|-|\text{min edge cover}|$

$|\text{max independent set}|=|V|-|\text{min vertex cover}|$
\section{Flow-Matching}
	\subsection{HopcroftKarp}
	\lstinputlisting[firstline=2]{4_Flow_Matching/Hopcroft-Karp.cpp}
	\subsection{KM}
	\lstinputlisting[firstline=2]{4_Flow_Matching/Kuhn_Munkres.cpp}
	\subsection{MCMF}
	\lstinputlisting[firstline=2]{4_Flow_Matching/MincostMaxflow_dijkstra.cpp}
	\subsection{GeneralGraphMatching}
	\lstinputlisting[firstline=2]{4_Flow_Matching/Maximum_Simple_Graph_Matching.cpp}
	\subsection{MaxWeightMaching}
	\lstinputlisting[firstline=2]{4_Flow_Matching/Maximum_Weight_Matching.cpp}
	\subsection{GlobalMinCut}
	\lstinputlisting[firstline=2]{4_Flow_Matching/SW-mincut.cpp}
	\subsection{BoundedFlow(Dinic)}
	\lstinputlisting[firstline=2]{4_Flow_Matching/BoundedFlow.cpp}
	\subsection{GomoryHuTree}
	\lstinputlisting[firstline=2]{4_Flow_Matching/Gomory_Hu_tree.cpp}
	\subsection{MinCostCirculation}
	\lstinputlisting[firstline=2]{4_Flow_Matching/MinCostCirculation.cpp}
	\subsection{FlowModelsBuilding}
	\input{4_Flow_Matching/Model.tex}
\section{Data Struture}
	\subsection{Treap}
	\lstinputlisting[firstline=2]{3_Data_Structure/Treap.cpp}
	\subsection{LinkCutTree}
	\lstinputlisting[firstline=2]{3_Data_Structure/link_cut_tree.cpp}
\section{String}
	\subsection{KMP}
	\lstinputlisting[firstline=2]{5_String/KMP.cpp}
	\subsection{Z}
	\lstinputlisting[firstline=2]{5_String/Z-value.cpp}
	\subsection{Manacher}
	\lstinputlisting[firstline=2]{5_String/Manacher.cpp}
	\subsection{SuffixArray}
	\lstinputlisting[firstline=2]{5_String/Suffix_Array.cpp}
	\subsection{SAIS}
	\lstinputlisting[firstline=2]{5_String/SAIS.cpp}
	\subsection{ACAutomaton}
	\lstinputlisting[firstline=2]{5_String/Aho-Corasick_Automatan.cpp}
	\subsection{MinRotation}
	\lstinputlisting[firstline=2]{5_String/Smallest_Rotation.cpp}
	\subsection{ExtSAM}
	\lstinputlisting[firstline=2]{5_String/exSAM.cpp}
	\subsection{PalindromeTree}
	\lstinputlisting[firstline=2]{5_String/PalTree.cpp}
\section{Number Theory}
	\subsection{Primes}
	12721 13331 14341 75577 123457 222557 556679 999983 1097774749 1076767633 100102021 999997771 1001010013 1000512343 987654361 999991231 999888733 98789101 987777733 999991921 1010101333 1010102101 1000000000039 1000000000000037 2305843009213693951 4611686018427387847 9223372036854775783 18446744073709551557
	\subsection{ExtGCD}
	\lstinputlisting[firstline=2]{6_Math/ExtGCD.cpp}
	\subsection{FloorCeil}
	\lstinputlisting[firstline=2]{6_Math/floor_ceil.cpp}
	\subsection{FloorSum}
	Computes
  $$ \begin{aligned}
    f(a, b, c, n) &= \sum^{n}_{i=0} \left\lfloor \frac{a \cdot i + b}{m} \right\rfloor \\
  \end{aligned} $$

Furthermore, Let $m = \left\lfloor\frac{an + b}{c}\right\rfloor$:
  $$ \begin{aligned}
    g(a, b, c, n) &= \sum_{i = 0}^{n}i\left\lfloor\frac{ai + b}{c}\right\rfloor \\
    &= \begin{cases}
        \left\lfloor{\frac{a}{c}}\right\rfloor \cdot \frac{n(n + 1)(2n + 1)}{6} + \left\lfloor\frac{b}{c}\right\rfloor \cdot \frac{n(n + 1)}{2} \\ + g(a\text{ mod } c, b\text{ mod } c, c, n), & a \geq c \lor b \geq c \\
      0, & n < 0 \lor a = 0 \\
      \frac{1}{2} \cdot (n(n + 1)m - f(c, c - b - 1, a, m - 1) \\ - h(c, c - b - 1, a, m - 1)), & \text{otherwise}
    \end{cases}
  \end{aligned} $$
  $$ \begin{aligned}
    h(a, b, c, n) &= \sum_{i = 0}^{n}\left\lfloor\frac{ai + b}{c}\right\rfloor^2 \\
    &= \begin{cases}
        \left\lfloor\frac{a}{c}\right\rfloor^2 \cdot \frac{n(n + 1)(2n + 1)}{6} + \left\lfloor\frac{b}{c}\right\rfloor^2 \cdot (n + 1) \\ + \left\lfloor\frac{a}{c}\right\rfloor \cdot \left\lfloor\frac{b}{c}\right\rfloor \cdot n(n + 1) \\ + h(a\text{ mod } c, b\text{ mod } c, c, n) \\ + 2\left\lfloor\frac{a}{c}\right\rfloor \cdot g(a\text{ mod } c, b\text{ mod } c, c, n) \\ + 2\left\lfloor\frac{b}{c}\right\rfloor \cdot f(a\text{ mod } c, b\text{ mod } c, c, n), & a \geq c \lor b \geq c \\
      0, & n < 0 \lor a = 0 \\
      nm(m + 1) - 2g(c, c - b - 1, a, m - 1) \\ - 2f(c, c - b - 1, a, m - 1) - f(a, b, c, n), & \text{otherwise}
    \end{cases}
  \end{aligned} $$
  
  
	\lstinputlisting[firstline=2]{6_Math/FloorSum.cpp}
	\subsection{MillerRabin}
	\lstinputlisting[firstline=2]{6_Math/Miller_Rabin.cpp}
	\subsection{PollardRho}
	\lstinputlisting[firstline=2]{6_Math/Pollard_Rho.cpp}
	\subsection{Fraction}
	\lstinputlisting[firstline=2]{6_Math/Fraction.cpp}
	\subsection{ChineseRemainder}
	\lstinputlisting[firstline=2]{6_Math/chineseRemainder.cpp}
	\subsection{FactorialMod$p^k$}
	\lstinputlisting[firstline=2]{6_Math/fac_no_p.cpp}
	\subsection{QuadraticResidue}
	\lstinputlisting[firstline=2]{6_Math/QuadraticResidue.cpp}
	\subsection{MeisselLehmer}
	\lstinputlisting[firstline=2]{6_Math/PiCount.cpp}
	\subsection{DiscreteLog}
	\lstinputlisting[firstline=2]{6_Math/DiscreteLog.cpp}
	\subsection{Theorems}
	\begin{itemize}
\item \textbf{Cramer's Rule}
$$
\begin{aligned}ax+by=e\\cx+dy=f\end{aligned}
\Rightarrow
\begin{aligned}x=\dfrac{ed-bf}{ad-bc}\\y=\dfrac{af-ec}{ad-bc}\end{aligned}
$$

\item \textbf{Vandermonde's Identity}
$$
C(n + m, k) = \sum_{i=0}^k C(n, i)C(m, k - i)
$$

\item \textbf{Kirchhoff's Theorem}

Denote $L$ be a $n \times n$ matrix as the Laplacian matrix of graph $G$, where $L_{ii} = d(i)$, $L_{ij} = -c$ where $c$ is the number of edge $(i, j)$ in $G$.
\begin{itemize}
    %\itemsep-0.5em
    \item The number of undirected spanning in $G$ is $\lvert \det(\tilde{L}_{11}) \rvert$.
    \item The number of directed spanning tree rooted at $r$ in $G$ is $\lvert \det(\tilde{L}_{rr}) \rvert$.
\end{itemize}

\item \textbf{Tutte's Matrix}

Let $D$ be a $n \times n$ matrix, where $d_{ij} = x_{ij}$ ($x_{ij}$ is chosen uniformly at random) if $i < j$ and $(i, j) \in E$, otherwise $d_{ij} = -d_{ji}$. $\frac{rank(D)}{2}$ is the maximum matching on $G$.

\item \textbf{Cayley's Formula}

\begin{itemize}
    %\itemsep-0.5em
  \item Given a degree sequence $d_1, d_2, \ldots, d_n$ for each \textit{labeled} vertices, there are $\frac{(n - 2)!}{(d_1 - 1)!(d_2 - 1)!\cdots(d_n - 1)!}$ spanning trees.
  \item Let $T_{n, k}$ be the number of \textit{labeled} forests on $n$ vertices with $k$ components, such that vertex $1, 2, \ldots, k$ belong to different components. Then $T_{n, k} = kn^{n - k - 1}$.
\end{itemize}

\item \textbf{Erdős-Gallai Theorem}

A sequence of nonnegative integers $d_1\ge\cdots\ge d_n$ can be represented as the degree sequence of a finite simple graph on $n$ vertices if and only if $d_1+\cdots+d_n$ is even and $\displaystyle\sum_{i-1}^kd_i\le k(k-1)+\displaystyle\sum_{i=k+1}^n\min(d_i,k)$ holds for every $1\le k\le n$.

\item \textbf{Gale-Ryser Theorem} 

A pair of sequences of nonnegative integers $a_1\ge\cdots\ge a_n$ and $b_1,\ldots,b_n$ is bigraphic (degree seqence of bipartie graph) if and only if $\displaystyle\sum_{i=1}^n a_i=\displaystyle\sum_{i=1}^n b_i$ and $\displaystyle\sum_{i=1}^k a_i\le \displaystyle\sum_{i=1}^n\min(b_i,k)$ holds for every $1\le k\le n$.

\item \textbf{Fulkerson-Chen-Anstee Theorem}

A sequence $(a_1,b_1),\ldots,(a_n,b_n)$ of nonnegative integer pairs with $a_1\ge\cdots\ge a_n$ is digraphic (in, out degree of a directed graph) if and only if $\displaystyle\sum_{i=1}^n a_i=\displaystyle\sum_{i=1}^n b_i$ and $\displaystyle\sum_{i=1}^k a_i\le \displaystyle\sum_{i=1}^k\min(b_i,k-1)+\displaystyle\sum_{i=k+1}^n\min(b_i,k)$ holds for every $1\le k\le n$.

\item \textbf{Möbius Inversion Formula}
\begin{itemize}
    %\itemsep-0.5em
  \item $f(n)=\sum_{d\mid n}g(d)\Leftrightarrow g(n)=\sum_{d\mid n}\mu(d)f(\frac{n}{d})$
  \item $f(n)=\sum_{n\mid d}g(d)\Leftrightarrow g(n)=\sum_{n\mid d}\mu(\frac{d}{n})f(d)$
\end{itemize}

\item \textbf{Lagrange Multiplier}

\begin{itemize}
    %\itemsep-0.5em
  \item Optimize $f(x_1, \ldots, x_n)$ when $k$ constraints $g_i(x_1, \ldots, x_n)=0$.
  \item Lagrangian function $\mathcal{L}(x_1, \ldots, x_n, \lambda_1, \ldots, \lambda_k) = f(x_1, \ldots, x_n) = \sum^{k}_{i=1}\lambda_i g_i(x_1, \ldots, x_n)$.
  \item The solution corresponding to the original constrained optimization is always a saddle point of the Lagrangian function.
\end{itemize}
\end{itemize}

	\subsection{Estimation}
	\begin{itemize}
\item Estimation

\begin{itemize}
    %\itemsep-0.5em
    \item Number of divisors: $100$ for $n < 5\cdot10^4$; $500$ for $n<10^7$; $2000$ for $n<10^{10}$, $200000$;$n<10^{19}$.
    \item Unordered integer partition: $1, 1, 2, 3, 5, 7, 11, 15, 22, 30$ for $n=0\sim 9$; $627$ for $n=20$; $\sim 2\cdot10^5$ for $n=50$; $\sim 2\cdot10^8$ for $n=100$.
    \item Ways of partitions of $n$ distinct elements: $B(n)=1, 1, 2, 5, 15, 52, 203, 877, 4140, 21147, 115975, 678570, 4213597,$\\
    $27644437, 190899322, \ldots$.
\end{itemize}

\end{itemize}

	\subsection{Numbers}
	\begin{itemize}
\item Bernoulli numbers

$B_0-1,B_1^{\pm}=\pm\frac{1}{2},B_2=\frac{1}{6},B_3=0$

$\displaystyle\sum_{j=0}^m\binom{m+1}{j}B_j=0$, EGF is $B(x) = \frac{x}{e^x - 1}=\displaystyle\sum_{n=0}^\infty B_n\frac{x^n}{n!}$.

$S_m(n)=\displaystyle\sum_{k=1}^nk^m=\frac{1}{m+1}\sum_{k=0}^m\binom{m+1}{k}B^{+}_kn^{m+1-k}$

\item Stirling numbers of the second kind
Partitions of $n$ distinct elements into exactly $k$ groups. 

$S(n, k) = S(n - 1, k - 1) + kS(n - 1, k), S(n, 1) = S(n, n) = 1$

$S(n, k) = \frac{1}{k!}\sum_{i=0}^{k}(-1)^{k-i}{k \choose i}i^n$

$x^n     = \sum_{i=0}^{n} S(n, i) (x)_i$

\item Pentagonal number theorem

$\displaystyle\prod_{n=1}^{\infty}(1-x^n)=1+\sum_{k=1}^{\infty}(-1)^k\left(x^{k(3k+1)/2} + x^{k(3k-1)/2}\right)$

\item Catalan numbers

$C^{(k)}_n = \displaystyle \frac{1}{(k - 1)n + 1}\binom{kn}{n}$

$C^{(k)}(x) = 1 + x [C^{(k)}(x)]^k$

\item Eulerian numbers

Number of permutations $\pi \in S_n$ in which exactly $k$ elements are greater than the previous element. $k$ $j$:s s.t. $\pi(j)>\pi(j+1)$, $k+1$ $j$:s s.t. $\pi(j)\geq j$, $k$ $j$:s s.t. $\pi(j)>j$.

$E(n,k) = (n-k)E(n-1,k-1) + (k+1)E(n-1,k)$

$E(n,0) = E(n,n-1) = 1$

$E(n,k) = \sum_{j=0}^k(-1)^j\binom{n+1}{j}(k+1-j)^n$

\end{itemize}

	\subsection{GeneratingFunctions}
	\begin{itemize}
\item Ordinary Generating Function
$A(x) = \sum_{i\ge 0} a_ix^i$
\begin{itemize}
    %\itemsep-0.5em
    \item $A(rx)             \Rightarrow r^na_n$
    \item $A(x) + B(x)       \Rightarrow a_n + b_n$
    \item $A(x)B(x)          \Rightarrow \sum_{i=0}^{n} a_ib_{n-i}$
    \item $A(x)^k            \Rightarrow \sum_{i_1+i_2+\cdots+i_k=n} a_{i_1}a_{i_2}\ldots a_{i_k}$
    \item $xA(x)'            \Rightarrow na_n$
    \item $\frac{A(x)}{1-x}  \Rightarrow \sum_{i=0}^{n} a_i$
\end{itemize}
\item Exponential Generating Function
$A(x) = \sum_{i\ge 0} \frac{a_i}{i!}x_i$
\begin{itemize}
    %\itemsep-0.5em
    \item $A(x) + B(x)       \Rightarrow a_n + b_n$
    \item $A^{(k)}(x)        \Rightarrow a_{n+k}$
    \item $A(x)B(x)          \Rightarrow \sum_{i=0}^{n} \binom{n}{i}a_ib_{n-i}$
    \item $A(x)^k            \Rightarrow \sum_{i_1+i_2+\cdots+i_k=n} \binom{n}{i_1, i_2, \ldots, i_k}a_{i_1}a_{i_2}\ldots a_{i_k}$
    \item $xA(x)             \Rightarrow na_n$
\end{itemize}
\item Special Generating Function
\begin{itemize}
    %\itemsep-0.5em
    \item $(1+x)^n           = \sum_{i\ge 0} \binom{n}{i}x^i$
    \item $\frac{1}{(1-x)^n} = \sum_{i\ge 0} \binom{i}{n-1}x^i$
    \item $S_k = \sum_{x=1}^nx^k$: $S = \sum_{p=0}^\infty x^p = \frac{e^x - e^{x(n+1)}}{1-e^x}$
\end{itemize}
\end{itemize}

\section{Linear Algebra}
	\subsection{GuassianElimination}
	\lstinputlisting[firstline=2]{6_Math/Simultaneous_Equations.cpp}
	\subsection{BerlekampMassey}
	\lstinputlisting[firstline=2]{6_Math/Berlekamp-Massey.cpp}
	\subsection{Simplex}
	\input{6_Math/SimplexConstruction.tex}
	\lstinputlisting[firstline=2]{6_Math/Simplex.cpp}
\section{Polynomials}
	\subsection{NTT (FFT)}
	\begin{tabular}{rll}
    Mod & $g$ & Form \\
    65 537 & 3 & $2^{16}+1$ \\
    998 244 353 & 3 & $119\cdot 2^{23} + 1$ \\
    1 315 962 881 & 3 & $1255\cdot 2^{20} + 1$ \\
    1 711 276 033 & 29 & $51\cdot 2^{25} + 1$ \\
    9 223 372 036 737 335 297 & 3 & $549755813881\cdot 2^{24} + 1$ \\
\end{tabular}
	\lstinputlisting[firstline=2]{7_Polynomial/NTT.cpp}
	\subsection{FHWT}
	\lstinputlisting[firstline=2]{7_Polynomial/Fast_Walsh_Transform.cpp}
	\subsection{PolynomialOperations}
	\lstinputlisting[firstline=2]{7_Polynomial/Polynomial_Operation.cpp}
	\subsection{NewtonMethod+MiscGF}
	Given $F(x)$ where

$$ F(x) = \sum_{i=0}^{\infty}{\alpha_i(x - \beta)^i} $$

for $\beta$ being some constant. Polynomial $P$ such that $F(P) = 0$ can be found iteratively. Denote by $Q_k$ the polynomial such that $F(Q_k) = 0 \pmod {x^{2^k}}$, then

$$ Q_{k+1} = Q_k - \frac{F(Q_k)}{F^\prime(Q_k)} \pmod {x^{2^{k+1}}} $$

\begin{itemize}
    \item $A^{-1}$: $B_{k+1} = B_k(2-AB_k) \mod x^{2^{k+1}}$
    \item $\ln A$: $(\ln A)' = \frac{A'}{A}$
    \item $\exp A$: $B_{k+1} = B_k(1 + A - \ln B_k) \mod x^{2^{k+1}}$
    \item $\sqrt A$: $B_{k+1} = \frac 1 2 (B_k + AB_k^{-1}) \mod x^{2^{k+1}}$
\end{itemize}

\section{Geometry}
	\subsection{Basic}
	\lstinputlisting[firstline=2]{8_Geometry/GeometryDefault.cpp}
	\subsection{ConvexHull}
	\lstinputlisting[firstline=2]{8_Geometry/Convex_hull.cpp}
	\subsection{SortByAngle}
	\lstinputlisting[firstline=2]{8_Geometry/Polar_Angle_Sort.cpp}
	\subsection{Formulas}
	\begin{itemize}

\item \textbf{Rotation}

$$
M(\theta) =
    \begin{bmatrix}
        \cos\theta & -\sin\theta\\
        \sin\theta & \cos\theta
    \end{bmatrix}
$$

90 degree: $(x, y) = (Y - y, x)$

\item \textbf{Pick's theorem}

For simple integer-coordinate polygon, 
$$A=\text{B} + \frac{I}{2} - 1$$
Where $A$ is the area; $B, I$ is \#lattice points in the interior, on the boundary.

\item \textbf{Spherical Cap}

\begin{itemize}
    %\itemsep-0.5em
  \item A portion of a sphere cut off by a plane.
  \item $r$: sphere radius, $a$: radius of the base of the cap, $h$: height of the cap, $\theta$: $\arcsin(a/r)$.
  \item Volume $=\pi h^2(3r-h)/3=\pi h(3a^2+h^2)/6=\pi r^3(2+\cos\theta)(1-\cos\theta)^2/3$.
  \item Area $=2\pi rh=\pi(a^2+h^2)=2\pi r^2(1-\cos\theta)$.
\end{itemize}

\item \textbf{Nearest points of two skew lines}

\begin{itemize}
\item $\text{Line 1}: \boldsymbol{v}_1 = \boldsymbol{p}_1 + t_1\boldsymbol{d}_1$
\item $\text{Line 2}: \boldsymbol{v}_2 = \boldsymbol{p}_2 + t_2\boldsymbol{d}_2$
\item $\boldsymbol{n} = \boldsymbol{d}_1\times \boldsymbol{d}_2$
\item $\boldsymbol{n}_1 = \boldsymbol{d}_1 \times \boldsymbol{n}$
\item $\boldsymbol{n}_2 = \boldsymbol{d}_2 \times \boldsymbol{n}$
\item $\boldsymbol{c}_1 = \boldsymbol{p}_1 + \frac{(\boldsymbol{p}_2 - \boldsymbol{p}_1)\cdot\boldsymbol{n}_2}{\boldsymbol{d}_1\cdot\boldsymbol{n}_2}\boldsymbol{d}_1$
\item $\boldsymbol{c}_2 = \boldsymbol{p}_2 + \frac{(\boldsymbol{p}_1 - \boldsymbol{p}_2)\cdot\boldsymbol{n}_1}{\boldsymbol{d}_2\cdot\boldsymbol{n}_1}\boldsymbol{d}_2$
\end{itemize}
\end{itemize}

	\subsection{TriangleHearts}
	\lstinputlisting[firstline=2]{8_Geometry/Heart.cpp}
	\subsection{PointSegmentDist}
	\lstinputlisting[firstline=2]{8_Geometry/PointSegDist.cpp}
	\subsection{PointInCircle}
	\lstinputlisting[firstline=2]{8_Geometry/point_in_circle.cpp}
	\subsection{PointInConvex}
	\lstinputlisting[firstline=2]{8_Geometry/PointInConvex.cpp}
	\subsection{PointTangentConvex}
	\lstinputlisting[firstline=2]{8_Geometry/TangentPointToHull.cpp}
	\subsection{CircTangentCirc}
	\lstinputlisting[firstline=2]{8_Geometry/Tangent_line_of_two_circles.cpp}
	\subsection{LineCircleIntersect}
	\lstinputlisting[firstline=2]{8_Geometry/Intersection_of_line_and_circle.cpp}
	\subsection{LineConvexIntersect}
	\lstinputlisting[firstline=2]{8_Geometry/Intersection_of_line_and_convex.cpp}
	\subsection{CircIntersectCirc}
	\lstinputlisting[firstline=2]{8_Geometry/Intersection_of_two_circles.cpp}
	\subsection{PolyIntersectCirc}
	\lstinputlisting[firstline=2]{8_Geometry/Intersection_of_polygon_and_circle.cpp}
	\subsection{PolyUnion}
	\lstinputlisting[firstline=2]{8_Geometry/PolyUnion.cpp}
	\subsection{MinkowskiSum}
	\lstinputlisting[firstline=2]{8_Geometry/Minkowski_Sum.cpp}
	\subsection{MinMaxEnclosingRect}
	\lstinputlisting[firstline=2]{8_Geometry/minMaxEnclosingRectangle.cpp}
	\subsection{MinEnclosingCircle}
	\lstinputlisting[firstline=3]{8_Geometry/Minimum_Enclosing_Circle.cpp}
	\subsection{CircleCover}
	\lstinputlisting[firstline=3]{8_Geometry/CircleCover.cpp}
	\subsection{LineCmp}
	\lstinputlisting[firstline=2]{8_Geometry/LineCmp.cpp}
	\subsection{Trapezoidalization}
	\lstinputlisting[firstline=2]{8_Geometry/Trapezoidalization.cpp}
	\subsection{HalfPlaneIntersect}
	\lstinputlisting[firstline=3]{8_Geometry/Half_plane_intersection.cpp}
	\subsection{RotatingSweepLine}
	\lstinputlisting[firstline=2]{8_Geometry/rotatingSweepLine.cpp}
	\subsection{DelaunayTriangulation}
	\lstinputlisting[firstline=3]{8_Geometry/DelaunayTriangulation.cpp}
	\subsection{VonoroiDiagram}
	\lstinputlisting[firstline=3]{8_Geometry/Triangulation_Vonoroi.cpp}
\section{Misc}
	\subsection{MoAlgoWithModify}
	\lstinputlisting[firstline=2]{9_Else/Mos_Algorithm_With_modification.cpp}
	\subsection{MoAlgoOnTree}
	\lstinputlisting[firstline=2]{9_Else/Mos_Algorithm_On_Tree.cpp}
	\subsection{MoAlgoAdvanced}
	\input{9_Else/Mos_Algorithm.tex}
	\subsection{HilbertCurve}
	\lstinputlisting[firstline=2]{9_Else/HilbertCurve.cpp}
	\subsection{SternBrocotTree}
	\begin{itemize}
    \item Construction: Root $\frac 1 1$, left/right neighbor $\frac 0 1, \frac 1 0$, 
          each node is sum of last left/right neighbor: $\frac a b, \frac c d \rightarrow \frac{a+c}{b+d}$
    \item Property: Adjacent (mid-order DFS) $\frac a b, \frac c d \Rightarrow bc-ad = 1$.
    \item Search known $\frac p q$: keep L-R alternative. Each step can calcaulated in $O(1) \Rightarrow$ total $O(\log C)$.
    \item Search unknown $\frac p q$: keep L-R alternative. Each step can calcaulated in $O(\log C)$ checks $\Rightarrow$ total $O(\log^2 C)$ checks.
\end{itemize}
    
	\subsection{AllLCS}
	\lstinputlisting[firstline=2]{9_Else/All_LCS.cpp}
	\subsection{SimulatedAnnealing}
	\lstinputlisting[firstline=2]{9_Else/SimulatedAnnealing.cpp}
	\subsection{SMAWK}
	\lstinputlisting[firstline=2]{9_Else/SMAWK.cpp}
	\subsection{Python}
	\lstinputlisting[firstline=1]{11_Python/misc.py}
	\subsection{LineContainer}
	\lstinputlisting[firstline=2]{9_Else/LineContainer.cpp}


\end{document}
=======
\documentclass[a4paper,10pt,twocolumn,oneside]{article}
\setlength{\columnsep}{10pt}                                                                    %兩欄模式的間距
\setlength{\columnseprule}{0pt}                                                                %兩欄模式間格線粗細

\usepackage{amsthm}								%定義,例題
\usepackage{amssymb}
%\usepackage[margin=2cm]{geometry}
\usepackage{fontspec}								%設定字體
\usepackage{color}
\usepackage[x11names]{xcolor}
\usepackage{listings}								%顯示code用的
%\usepackage[Glenn]{fncychap}						%排版,頁面模板
\usepackage{fancyhdr}								%設定頁首頁尾
\usepackage{graphicx}								%Graphic
\usepackage{enumerate}
\usepackage{titlesec}
\usepackage{amsmath}
\usepackage[CheckSingle, CJKmath]{xeCJK}
% \usepackage{CJKulem}

%\usepackage[T1]{fontenc}
\usepackage{amsmath, courier, listings, fancyhdr, graphicx}
\topmargin=0pt
\headsep=5pt
\textheight=780pt
\footskip=0pt
\voffset=-40pt
\textwidth=545pt
\marginparsep=0pt
\marginparwidth=0pt
\marginparpush=0pt
\oddsidemargin=0pt
\evensidemargin=0pt
\hoffset=-42pt

\titlespacing\subsection{0pt}{4pt plus 2pt minus 2pt}{0pt plus 2pt minus 2pt}


%\renewcommand\listfigurename{圖目錄}
%\renewcommand\listtablename{表目錄} 

%%%%%%%%%%%%%%%%%%%%%%%%%%%%%

\setmainfont{Consolas}
\setmonofont{Consolas}
\setCJKmainfont{Noto Sans TC}
% \setmainfont{Ubuntu Mono}
% \setmonofont{Ubuntu Mono}
% \setCJKmainfont{Noto Sans CJK TC} % on linux

\XeTeXlinebreaklocale "zh"						%中文自動換行
\XeTeXlinebreakskip = 0pt plus 1pt				%設定段落之間的距離
\setcounter{secnumdepth}{3}						%目錄顯示第三層

%%%%%%%%%%%%%%%%%%%%%%%%%%%%%
\newcommand\digitstyle{\color{DarkOrchid3}}
\makeatletter
\lst@CCPutMacro\lst@ProcessOther {"2D}{\lst@ttfamily{-{}}{-{}}}
\@empty\z@\@empty

\newtoks\BBQube@token
\newcount\BBQube@length
\def\BBQube@ResetToken{\BBQube@token{}\BBQube@length\z@}
\def\BBQube@Append#1{\advance\BBQube@length\@ne
  \BBQube@token=\expandafter{\the\BBQube@token#1}}

\def\BBQube@ProcessChar#1{%
  \ifnum\lst@mode=\lst@Pmode%
    \ifnum 9<1#1%
      \expandafter\BBQube@Append{\begingroup\digitstyle #1 \endgroup}%
    \else%
      \expandafter\BBQube@Append{#1}%
    \fi%
  \else%
    \expandafter\BBQube@Append{#1}%
  \fi%
}
\def\BBQube@ProcessStringInner#1#2\BBQube@nil{%
  \expandafter\BBQube@ProcessChar{#1}%
  \if\relax\detokenize{#2}\relax%
  \else%
    \expandafter\BBQube@ProcessStringInner#2\BBQube@nil%
  \fi%
}

\def\BBQube@ProcessString#1{\expandafter\BBQube@ProcessStringInner#1\BBQube@nil}

\lst@AddToHook{OutputOther}{%
\BBQube@ResetToken%
\expandafter\BBQube@ProcessString{\the\lst@token}%
\lst@token=\expandafter{\the\BBQube@token}%
}
\makeatother
\lstset{											% Code顯示
language=C++,										% the language of the code
basicstyle=\footnotesize\ttfamily, 						% the size of the fonts that are used for the code
%numbers=left,										% where to put the line-numbers
numberstyle=\footnotesize,						% the size of the fonts that are used for the line-numbers
stepnumber=1,										% the step between two line-numbers. If it's 1, each line  will be numbered
numbersep=5pt,										% how far the line-numbers are from the code
backgroundcolor=\color{white},					% choose the background color. You must add \usepackage{color}
showspaces=false,									% show spaces adding particular underscores
showstringspaces=false,							% underline spaces within strings
showtabs=false,									% show tabs within strings adding particular underscores
frame=false,											% adds a frame around the code
tabsize=2,											% sets default tabsize to 2 spaces
captionpos=b,										% sets the caption-position to bottom
breaklines=true,									% sets automatic line breaking
breakatwhitespace=false,							% sets if automatic breaks should only happen at whitespace
escapeinside={\%*}{*)},							% if you want to add a comment within your code
morekeywords={constexpr},									% if you want to add more keywords to the set
keywordstyle=\bfseries\color{Blue1},
commentstyle=\itshape\color{Red4},
stringstyle=\itshape\color{Green4},
}

%%%%%%%%%%%%%%%%%%%%%%%%%%%%%

\begin{document}
\pagestyle{fancy}
\fancyfoot{}
%\fancyfoot[R]{\includegraphics[width=20pt]{ironwood.jpg}}
\fancyhead[L]{National Taiwan University RngBased (forked from 8BQube)}
\fancyhead[R]{\thepage}
\renewcommand{\headrulewidth}{0.4pt}
\renewcommand{\contentsname}{Contents} 

\scriptsize
\tableofcontents
%%%%%%%%%%%%%%%%%%%%%%%%%%%%%

%\newpage

\section{Basic}
	\subsection{.vimrc}
	\lstinputlisting[firstline=1]{1_Basic/.vimrc}
	\subsection{PBDS}
	\lstinputlisting[firstline=2]{1_Basic/PBDS.cpp}
	\subsection{pragma}
	\lstinputlisting[firstline=2]{1_Basic/pragma.cpp}
	\subsection{LambdaCompare}
	\lstinputlisting[firstline=2]{1_Basic/lambda_cmp.cpp}
\section{Graph}
	\subsection{2SAT/SCC}
	\lstinputlisting[firstline=2]{2_Graph/2SAT.cpp}
	\subsection{BCC Vertex}
	\lstinputlisting[firstline=2]{2_Graph/BCC_Vertex.cpp}
	\subsection{VirtualTree}
	\lstinputlisting[firstline=2]{2_Graph/Virtual_Tree.cpp}
	\subsection{MinimumMeanCycle}
	\lstinputlisting[firstline=1]{2_Graph/MinimumMeanCycle.cpp}
	\subsection{MaximumCliqueDyn}
	\lstinputlisting[firstline=2]{2_Graph/Maximum_Clique_Dyn.cpp}
	\subsection{DominatorTree}
	\lstinputlisting[firstline=2]{2_Graph/Dominator_Tree.cpp}
	\subsection{DMST(slow)}
	\lstinputlisting[firstline=2]{2_Graph/Minimum_Arborescence.cpp}
	\subsection{DMST}
	\lstinputlisting[firstline=2]{2_Graph/DMST.cpp}
	\subsection{VizingTheorem}
	\lstinputlisting[firstline=2]{2_Graph/Vizing.cpp}
	\subsection{MinimumCliqueCover}
	\lstinputlisting[firstline=3]{2_Graph/Minimum_Clique_Cover.cpp}
	\subsection{CountMaximalClique}
	\lstinputlisting[firstline=2]{2_Graph/NumberofMaximalClique.cpp}
	\subsection{Theorems}
	$|\text{max independent edge set}|=|V|-|\text{min edge cover}|$

$|\text{max independent set}|=|V|-|\text{min vertex cover}|$
\section{Flow-Matching}
	\subsection{HopcroftKarp}
	\lstinputlisting[firstline=2]{4_Flow_Matching/Hopcroft-Karp.cpp}
	\subsection{KM}
	\lstinputlisting[firstline=2]{4_Flow_Matching/Kuhn_Munkres.cpp}
	\subsection{MCMF}
	\lstinputlisting[firstline=2]{4_Flow_Matching/MincostMaxflow_dijkstra.cpp}
	\subsection{GeneralGraphMatching}
	\lstinputlisting[firstline=2]{4_Flow_Matching/Maximum_Simple_Graph_Matching.cpp}
	\subsection{MaxWeightMaching}
	\lstinputlisting[firstline=2]{4_Flow_Matching/Maximum_Weight_Matching.cpp}
	\subsection{GlobalMinCut}
	\lstinputlisting[firstline=2]{4_Flow_Matching/SW-mincut.cpp}
	\subsection{BoundedFlow(Dinic)}
	\lstinputlisting[firstline=2]{4_Flow_Matching/BoundedFlow.cpp}
	\subsection{GomoryHuTree}
	\lstinputlisting[firstline=2]{4_Flow_Matching/Gomory_Hu_tree.cpp}
	\subsection{MinCostCirculation}
	\lstinputlisting[firstline=2]{4_Flow_Matching/MinCostCirculation.cpp}
	\subsection{FlowModelsBuilding}
	\input{4_Flow_Matching/Model.tex}
\section{Data Struture}
	\subsection{Treap}
	\lstinputlisting[firstline=2]{3_Data_Structure/Treap.cpp}
	\subsection{LinkCutTree}
	\lstinputlisting[firstline=2]{3_Data_Structure/link_cut_tree.cpp}
\section{String}
	\subsection{KMP}
	\lstinputlisting[firstline=2]{5_String/KMP.cpp}
	\subsection{Z}
	\lstinputlisting[firstline=2]{5_String/Z-value.cpp}
	\subsection{Manacher}
	\lstinputlisting[firstline=2]{5_String/Manacher.cpp}
	\subsection{SuffixArray}
	\lstinputlisting[firstline=2]{5_String/Suffix_Array.cpp}
	\subsection{SAIS}
	\lstinputlisting[firstline=2]{5_String/SAIS.cpp}
	\subsection{ACAutomaton}
	\lstinputlisting[firstline=2]{5_String/Aho-Corasick_Automatan.cpp}
	\subsection{MinRotation}
	\lstinputlisting[firstline=2]{5_String/Smallest_Rotation.cpp}
	\subsection{ExtSAM}
	\lstinputlisting[firstline=2]{5_String/exSAM.cpp}
	\subsection{PalindromeTree}
	\lstinputlisting[firstline=2]{5_String/PalTree.cpp}
\section{Number Theory}
	\subsection{Primes}
	12721 13331 14341 75577 123457 222557 556679 999983 1097774749 1076767633 100102021 999997771 1001010013 1000512343 987654361 999991231 999888733 98789101 987777733 999991921 1010101333 1010102101 1000000000039 1000000000000037 2305843009213693951 4611686018427387847 9223372036854775783 18446744073709551557
	\subsection{ExtGCD}
	\lstinputlisting[firstline=2]{6_Math/ExtGCD.cpp}
	\subsection{FloorCeil}
	\lstinputlisting[firstline=2]{6_Math/floor_ceil.cpp}
	\subsection{FloorSum}
	Computes
  $$ \begin{aligned}
    f(a, b, c, n) &= \sum^{n}_{i=0} \left\lfloor \frac{a \cdot i + b}{m} \right\rfloor \\
  \end{aligned} $$

Furthermore, Let $m = \left\lfloor\frac{an + b}{c}\right\rfloor$:
  $$ \begin{aligned}
    g(a, b, c, n) &= \sum_{i = 0}^{n}i\left\lfloor\frac{ai + b}{c}\right\rfloor \\
    &= \begin{cases}
        \left\lfloor{\frac{a}{c}}\right\rfloor \cdot \frac{n(n + 1)(2n + 1)}{6} + \left\lfloor\frac{b}{c}\right\rfloor \cdot \frac{n(n + 1)}{2} \\ + g(a\text{ mod } c, b\text{ mod } c, c, n), & a \geq c \lor b \geq c \\
      0, & n < 0 \lor a = 0 \\
      \frac{1}{2} \cdot (n(n + 1)m - f(c, c - b - 1, a, m - 1) \\ - h(c, c - b - 1, a, m - 1)), & \text{otherwise}
    \end{cases}
  \end{aligned} $$
  $$ \begin{aligned}
    h(a, b, c, n) &= \sum_{i = 0}^{n}\left\lfloor\frac{ai + b}{c}\right\rfloor^2 \\
    &= \begin{cases}
        \left\lfloor\frac{a}{c}\right\rfloor^2 \cdot \frac{n(n + 1)(2n + 1)}{6} + \left\lfloor\frac{b}{c}\right\rfloor^2 \cdot (n + 1) \\ + \left\lfloor\frac{a}{c}\right\rfloor \cdot \left\lfloor\frac{b}{c}\right\rfloor \cdot n(n + 1) \\ + h(a\text{ mod } c, b\text{ mod } c, c, n) \\ + 2\left\lfloor\frac{a}{c}\right\rfloor \cdot g(a\text{ mod } c, b\text{ mod } c, c, n) \\ + 2\left\lfloor\frac{b}{c}\right\rfloor \cdot f(a\text{ mod } c, b\text{ mod } c, c, n), & a \geq c \lor b \geq c \\
      0, & n < 0 \lor a = 0 \\
      nm(m + 1) - 2g(c, c - b - 1, a, m - 1) \\ - 2f(c, c - b - 1, a, m - 1) - f(a, b, c, n), & \text{otherwise}
    \end{cases}
  \end{aligned} $$
  
  
	\lstinputlisting[firstline=2]{6_Math/FloorSum.cpp}
	\subsection{MillerRabin}
	\lstinputlisting[firstline=2]{6_Math/Miller_Rabin.cpp}
	\subsection{PollardRho}
	\lstinputlisting[firstline=2]{6_Math/Pollard_Rho.cpp}
	\subsection{Fraction}
	\lstinputlisting[firstline=2]{6_Math/Fraction.cpp}
	\subsection{ChineseRemainder}
	\lstinputlisting[firstline=2]{6_Math/chineseRemainder.cpp}
	\subsection{FactorialMod$p^k$}
	\lstinputlisting[firstline=2]{6_Math/fac_no_p.cpp}
	\subsection{QuadraticResidue}
	\lstinputlisting[firstline=2]{6_Math/QuadraticResidue.cpp}
	\subsection{MeisselLehmer}
	\lstinputlisting[firstline=2]{6_Math/PiCount.cpp}
	\subsection{DiscreteLog}
	\lstinputlisting[firstline=2]{6_Math/DiscreteLog.cpp}
	\subsection{Theorems}
	\begin{itemize}
\item \textbf{Cramer's Rule}
$$
\begin{aligned}ax+by=e\\cx+dy=f\end{aligned}
\Rightarrow
\begin{aligned}x=\dfrac{ed-bf}{ad-bc}\\y=\dfrac{af-ec}{ad-bc}\end{aligned}
$$

\item \textbf{Vandermonde's Identity}
$$
C(n + m, k) = \sum_{i=0}^k C(n, i)C(m, k - i)
$$

\item \textbf{Kirchhoff's Theorem}

Denote $L$ be a $n \times n$ matrix as the Laplacian matrix of graph $G$, where $L_{ii} = d(i)$, $L_{ij} = -c$ where $c$ is the number of edge $(i, j)$ in $G$.
\begin{itemize}
    %\itemsep-0.5em
    \item The number of undirected spanning in $G$ is $\lvert \det(\tilde{L}_{11}) \rvert$.
    \item The number of directed spanning tree rooted at $r$ in $G$ is $\lvert \det(\tilde{L}_{rr}) \rvert$.
\end{itemize}

\item \textbf{Tutte's Matrix}

Let $D$ be a $n \times n$ matrix, where $d_{ij} = x_{ij}$ ($x_{ij}$ is chosen uniformly at random) if $i < j$ and $(i, j) \in E$, otherwise $d_{ij} = -d_{ji}$. $\frac{rank(D)}{2}$ is the maximum matching on $G$.

\item \textbf{Cayley's Formula}

\begin{itemize}
    %\itemsep-0.5em
  \item Given a degree sequence $d_1, d_2, \ldots, d_n$ for each \textit{labeled} vertices, there are $\frac{(n - 2)!}{(d_1 - 1)!(d_2 - 1)!\cdots(d_n - 1)!}$ spanning trees.
  \item Let $T_{n, k}$ be the number of \textit{labeled} forests on $n$ vertices with $k$ components, such that vertex $1, 2, \ldots, k$ belong to different components. Then $T_{n, k} = kn^{n - k - 1}$.
\end{itemize}

\item \textbf{Erdős-Gallai Theorem}

A sequence of nonnegative integers $d_1\ge\cdots\ge d_n$ can be represented as the degree sequence of a finite simple graph on $n$ vertices if and only if $d_1+\cdots+d_n$ is even and $\displaystyle\sum_{i-1}^kd_i\le k(k-1)+\displaystyle\sum_{i=k+1}^n\min(d_i,k)$ holds for every $1\le k\le n$.

\item \textbf{Gale-Ryser Theorem} 

A pair of sequences of nonnegative integers $a_1\ge\cdots\ge a_n$ and $b_1,\ldots,b_n$ is bigraphic (degree seqence of bipartie graph) if and only if $\displaystyle\sum_{i=1}^n a_i=\displaystyle\sum_{i=1}^n b_i$ and $\displaystyle\sum_{i=1}^k a_i\le \displaystyle\sum_{i=1}^n\min(b_i,k)$ holds for every $1\le k\le n$.

\item \textbf{Fulkerson-Chen-Anstee Theorem}

A sequence $(a_1,b_1),\ldots,(a_n,b_n)$ of nonnegative integer pairs with $a_1\ge\cdots\ge a_n$ is digraphic (in, out degree of a directed graph) if and only if $\displaystyle\sum_{i=1}^n a_i=\displaystyle\sum_{i=1}^n b_i$ and $\displaystyle\sum_{i=1}^k a_i\le \displaystyle\sum_{i=1}^k\min(b_i,k-1)+\displaystyle\sum_{i=k+1}^n\min(b_i,k)$ holds for every $1\le k\le n$.

\item \textbf{Möbius Inversion Formula}
\begin{itemize}
    %\itemsep-0.5em
  \item $f(n)=\sum_{d\mid n}g(d)\Leftrightarrow g(n)=\sum_{d\mid n}\mu(d)f(\frac{n}{d})$
  \item $f(n)=\sum_{n\mid d}g(d)\Leftrightarrow g(n)=\sum_{n\mid d}\mu(\frac{d}{n})f(d)$
\end{itemize}

\item \textbf{Lagrange Multiplier}

\begin{itemize}
    %\itemsep-0.5em
  \item Optimize $f(x_1, \ldots, x_n)$ when $k$ constraints $g_i(x_1, \ldots, x_n)=0$.
  \item Lagrangian function $\mathcal{L}(x_1, \ldots, x_n, \lambda_1, \ldots, \lambda_k) = f(x_1, \ldots, x_n) = \sum^{k}_{i=1}\lambda_i g_i(x_1, \ldots, x_n)$.
  \item The solution corresponding to the original constrained optimization is always a saddle point of the Lagrangian function.
\end{itemize}
\end{itemize}

	\subsection{Estimation}
	\begin{itemize}
\item Estimation

\begin{itemize}
    %\itemsep-0.5em
    \item Number of divisors: $100$ for $n < 5\cdot10^4$; $500$ for $n<10^7$; $2000$ for $n<10^{10}$, $200000$;$n<10^{19}$.
    \item Unordered integer partition: $1, 1, 2, 3, 5, 7, 11, 15, 22, 30$ for $n=0\sim 9$; $627$ for $n=20$; $\sim 2\cdot10^5$ for $n=50$; $\sim 2\cdot10^8$ for $n=100$.
    \item Ways of partitions of $n$ distinct elements: $B(n)=1, 1, 2, 5, 15, 52, 203, 877, 4140, 21147, 115975, 678570, 4213597,$\\
    $27644437, 190899322, \ldots$.
\end{itemize}

\end{itemize}

	\subsection{Numbers}
	\begin{itemize}
\item Bernoulli numbers

$B_0-1,B_1^{\pm}=\pm\frac{1}{2},B_2=\frac{1}{6},B_3=0$

$\displaystyle\sum_{j=0}^m\binom{m+1}{j}B_j=0$, EGF is $B(x) = \frac{x}{e^x - 1}=\displaystyle\sum_{n=0}^\infty B_n\frac{x^n}{n!}$.

$S_m(n)=\displaystyle\sum_{k=1}^nk^m=\frac{1}{m+1}\sum_{k=0}^m\binom{m+1}{k}B^{+}_kn^{m+1-k}$

\item Stirling numbers of the second kind
Partitions of $n$ distinct elements into exactly $k$ groups. 

$S(n, k) = S(n - 1, k - 1) + kS(n - 1, k), S(n, 1) = S(n, n) = 1$

$S(n, k) = \frac{1}{k!}\sum_{i=0}^{k}(-1)^{k-i}{k \choose i}i^n$

$x^n     = \sum_{i=0}^{n} S(n, i) (x)_i$

\item Pentagonal number theorem

$\displaystyle\prod_{n=1}^{\infty}(1-x^n)=1+\sum_{k=1}^{\infty}(-1)^k\left(x^{k(3k+1)/2} + x^{k(3k-1)/2}\right)$

\item Catalan numbers

$C^{(k)}_n = \displaystyle \frac{1}{(k - 1)n + 1}\binom{kn}{n}$

$C^{(k)}(x) = 1 + x [C^{(k)}(x)]^k$

\item Eulerian numbers

Number of permutations $\pi \in S_n$ in which exactly $k$ elements are greater than the previous element. $k$ $j$:s s.t. $\pi(j)>\pi(j+1)$, $k+1$ $j$:s s.t. $\pi(j)\geq j$, $k$ $j$:s s.t. $\pi(j)>j$.

$E(n,k) = (n-k)E(n-1,k-1) + (k+1)E(n-1,k)$

$E(n,0) = E(n,n-1) = 1$

$E(n,k) = \sum_{j=0}^k(-1)^j\binom{n+1}{j}(k+1-j)^n$

\end{itemize}

	\subsection{GeneratingFunctions}
	\begin{itemize}
\item Ordinary Generating Function
$A(x) = \sum_{i\ge 0} a_ix^i$
\begin{itemize}
    %\itemsep-0.5em
    \item $A(rx)             \Rightarrow r^na_n$
    \item $A(x) + B(x)       \Rightarrow a_n + b_n$
    \item $A(x)B(x)          \Rightarrow \sum_{i=0}^{n} a_ib_{n-i}$
    \item $A(x)^k            \Rightarrow \sum_{i_1+i_2+\cdots+i_k=n} a_{i_1}a_{i_2}\ldots a_{i_k}$
    \item $xA(x)'            \Rightarrow na_n$
    \item $\frac{A(x)}{1-x}  \Rightarrow \sum_{i=0}^{n} a_i$
\end{itemize}
\item Exponential Generating Function
$A(x) = \sum_{i\ge 0} \frac{a_i}{i!}x_i$
\begin{itemize}
    %\itemsep-0.5em
    \item $A(x) + B(x)       \Rightarrow a_n + b_n$
    \item $A^{(k)}(x)        \Rightarrow a_{n+k}$
    \item $A(x)B(x)          \Rightarrow \sum_{i=0}^{n} \binom{n}{i}a_ib_{n-i}$
    \item $A(x)^k            \Rightarrow \sum_{i_1+i_2+\cdots+i_k=n} \binom{n}{i_1, i_2, \ldots, i_k}a_{i_1}a_{i_2}\ldots a_{i_k}$
    \item $xA(x)             \Rightarrow na_n$
\end{itemize}
\item Special Generating Function
\begin{itemize}
    %\itemsep-0.5em
    \item $(1+x)^n           = \sum_{i\ge 0} \binom{n}{i}x^i$
    \item $\frac{1}{(1-x)^n} = \sum_{i\ge 0} \binom{i}{n-1}x^i$
    \item $S_k = \sum_{x=1}^nx^k$: $S = \sum_{p=0}^\infty x^p = \frac{e^x - e^{x(n+1)}}{1-e^x}$
\end{itemize}
\end{itemize}

\section{Linear Algebra}
	\subsection{GuassianElimination}
	\lstinputlisting[firstline=2]{6_Math/Simultaneous_Equations.cpp}
	\subsection{BerlekampMassey}
	\lstinputlisting[firstline=2]{6_Math/Berlekamp-Massey.cpp}
	\subsection{Simplex}
	\input{6_Math/SimplexConstruction.tex}
	\lstinputlisting[firstline=2]{6_Math/Simplex.cpp}
\section{Polynomials}
	\subsection{NTT (FFT)}
	\begin{tabular}{rll}
    Mod & $g$ & Form \\
    65 537 & 3 & $2^{16}+1$ \\
    998 244 353 & 3 & $119\cdot 2^{23} + 1$ \\
    1 315 962 881 & 3 & $1255\cdot 2^{20} + 1$ \\
    1 711 276 033 & 29 & $51\cdot 2^{25} + 1$ \\
    9 223 372 036 737 335 297 & 3 & $549755813881\cdot 2^{24} + 1$ \\
\end{tabular}
	\lstinputlisting[firstline=2]{7_Polynomial/NTT.cpp}
	\subsection{FHWT}
	\lstinputlisting[firstline=2]{7_Polynomial/Fast_Walsh_Transform.cpp}
	\subsection{PolynomialOperations}
	\lstinputlisting[firstline=2]{7_Polynomial/Polynomial_Operation.cpp}
	\subsection{NewtonMethod+MiscGF}
	Given $F(x)$ where

$$ F(x) = \sum_{i=0}^{\infty}{\alpha_i(x - \beta)^i} $$

for $\beta$ being some constant. Polynomial $P$ such that $F(P) = 0$ can be found iteratively. Denote by $Q_k$ the polynomial such that $F(Q_k) = 0 \pmod {x^{2^k}}$, then

$$ Q_{k+1} = Q_k - \frac{F(Q_k)}{F^\prime(Q_k)} \pmod {x^{2^{k+1}}} $$

\begin{itemize}
    \item $A^{-1}$: $B_{k+1} = B_k(2-AB_k) \mod x^{2^{k+1}}$
    \item $\ln A$: $(\ln A)' = \frac{A'}{A}$
    \item $\exp A$: $B_{k+1} = B_k(1 + A - \ln B_k) \mod x^{2^{k+1}}$
    \item $\sqrt A$: $B_{k+1} = \frac 1 2 (B_k + AB_k^{-1}) \mod x^{2^{k+1}}$
\end{itemize}

\section{Geometry}
	\subsection{Basic}
	\lstinputlisting[firstline=2]{8_Geometry/GeometryDefault.cpp}
	\subsection{ConvexHull}
	\lstinputlisting[firstline=2]{8_Geometry/Convex_hull.cpp}
	\subsection{SortByAngle}
	\lstinputlisting[firstline=2]{8_Geometry/Polar_Angle_Sort.cpp}
	\subsection{Formulas}
	\begin{itemize}

\item \textbf{Rotation}

$$
M(\theta) =
    \begin{bmatrix}
        \cos\theta & -\sin\theta\\
        \sin\theta & \cos\theta
    \end{bmatrix}
$$

90 degree: $(x, y) = (Y - y, x)$

\item \textbf{Pick's theorem}

For simple integer-coordinate polygon, 
$$A=\text{B} + \frac{I}{2} - 1$$
Where $A$ is the area; $B, I$ is \#lattice points in the interior, on the boundary.

\item \textbf{Spherical Cap}

\begin{itemize}
    %\itemsep-0.5em
  \item A portion of a sphere cut off by a plane.
  \item $r$: sphere radius, $a$: radius of the base of the cap, $h$: height of the cap, $\theta$: $\arcsin(a/r)$.
  \item Volume $=\pi h^2(3r-h)/3=\pi h(3a^2+h^2)/6=\pi r^3(2+\cos\theta)(1-\cos\theta)^2/3$.
  \item Area $=2\pi rh=\pi(a^2+h^2)=2\pi r^2(1-\cos\theta)$.
\end{itemize}

\item \textbf{Nearest points of two skew lines}

\begin{itemize}
\item $\text{Line 1}: \boldsymbol{v}_1 = \boldsymbol{p}_1 + t_1\boldsymbol{d}_1$
\item $\text{Line 2}: \boldsymbol{v}_2 = \boldsymbol{p}_2 + t_2\boldsymbol{d}_2$
\item $\boldsymbol{n} = \boldsymbol{d}_1\times \boldsymbol{d}_2$
\item $\boldsymbol{n}_1 = \boldsymbol{d}_1 \times \boldsymbol{n}$
\item $\boldsymbol{n}_2 = \boldsymbol{d}_2 \times \boldsymbol{n}$
\item $\boldsymbol{c}_1 = \boldsymbol{p}_1 + \frac{(\boldsymbol{p}_2 - \boldsymbol{p}_1)\cdot\boldsymbol{n}_2}{\boldsymbol{d}_1\cdot\boldsymbol{n}_2}\boldsymbol{d}_1$
\item $\boldsymbol{c}_2 = \boldsymbol{p}_2 + \frac{(\boldsymbol{p}_1 - \boldsymbol{p}_2)\cdot\boldsymbol{n}_1}{\boldsymbol{d}_2\cdot\boldsymbol{n}_1}\boldsymbol{d}_2$
\end{itemize}
\end{itemize}

	\subsection{TriangleHearts}
	\lstinputlisting[firstline=2]{8_Geometry/Heart.cpp}
	\subsection{PointSegmentDist}
	\lstinputlisting[firstline=2]{8_Geometry/PointSegDist.cpp}
	\subsection{PointInCircle}
	\lstinputlisting[firstline=2]{8_Geometry/point_in_circle.cpp}
	\subsection{PointInConvex}
	\lstinputlisting[firstline=2]{8_Geometry/PointInConvex.cpp}
	\subsection{PointTangentConvex}
	\lstinputlisting[firstline=2]{8_Geometry/TangentPointToHull.cpp}
	\subsection{CircTangentCirc}
	\lstinputlisting[firstline=2]{8_Geometry/Tangent_line_of_two_circles.cpp}
	\subsection{LineCircleIntersect}
	\lstinputlisting[firstline=2]{8_Geometry/Intersection_of_line_and_circle.cpp}
	\subsection{LineConvexIntersect}
	\lstinputlisting[firstline=2]{8_Geometry/Intersection_of_line_and_convex.cpp}
	\subsection{CircIntersectCirc}
	\lstinputlisting[firstline=2]{8_Geometry/Intersection_of_two_circles.cpp}
	\subsection{PolyIntersectCirc}
	\lstinputlisting[firstline=2]{8_Geometry/Intersection_of_polygon_and_circle.cpp}
	\subsection{PolyUnion}
	\lstinputlisting[firstline=2]{8_Geometry/PolyUnion.cpp}
	\subsection{MinkowskiSum}
	\lstinputlisting[firstline=2]{8_Geometry/Minkowski_Sum.cpp}
	\subsection{MinMaxEnclosingRect}
	\lstinputlisting[firstline=2]{8_Geometry/minMaxEnclosingRectangle.cpp}
	\subsection{MinEnclosingCircle}
	\lstinputlisting[firstline=3]{8_Geometry/Minimum_Enclosing_Circle.cpp}
	\subsection{CircleCover}
	\lstinputlisting[firstline=3]{8_Geometry/CircleCover.cpp}
	\subsection{LineCmp}
	\lstinputlisting[firstline=2]{8_Geometry/LineCmp.cpp}
	\subsection{Trapezoidalization}
	\lstinputlisting[firstline=2]{8_Geometry/Trapezoidalization.cpp}
	\subsection{HalfPlaneIntersect}
	\lstinputlisting[firstline=3]{8_Geometry/Half_plane_intersection.cpp}
	\subsection{RotatingSweepLine}
	\lstinputlisting[firstline=2]{8_Geometry/rotatingSweepLine.cpp}
	\subsection{DelaunayTriangulation}
	\lstinputlisting[firstline=3]{8_Geometry/DelaunayTriangulation.cpp}
	\subsection{VonoroiDiagram}
	\lstinputlisting[firstline=3]{8_Geometry/Triangulation_Vonoroi.cpp}
\section{Misc}
	\subsection{MoAlgoWithModify}
	\lstinputlisting[firstline=2]{9_Else/Mos_Algorithm_With_modification.cpp}
	\subsection{MoAlgoOnTree}
	\lstinputlisting[firstline=2]{9_Else/Mos_Algorithm_On_Tree.cpp}
	\subsection{MoAlgoAdvanced}
	\input{9_Else/Mos_Algorithm.tex}
	\subsection{HilbertCurve}
	\lstinputlisting[firstline=2]{9_Else/HilbertCurve.cpp}
	\subsection{SternBrocotTree}
	\begin{itemize}
    \item Construction: Root $\frac 1 1$, left/right neighbor $\frac 0 1, \frac 1 0$, 
          each node is sum of last left/right neighbor: $\frac a b, \frac c d \rightarrow \frac{a+c}{b+d}$
    \item Property: Adjacent (mid-order DFS) $\frac a b, \frac c d \Rightarrow bc-ad = 1$.
    \item Search known $\frac p q$: keep L-R alternative. Each step can calcaulated in $O(1) \Rightarrow$ total $O(\log C)$.
    \item Search unknown $\frac p q$: keep L-R alternative. Each step can calcaulated in $O(\log C)$ checks $\Rightarrow$ total $O(\log^2 C)$ checks.
\end{itemize}
    
	\subsection{AllLCS}
	\lstinputlisting[firstline=2]{9_Else/All_LCS.cpp}
	\subsection{SimulatedAnnealing}
	\lstinputlisting[firstline=2]{9_Else/SimulatedAnnealing.cpp}
	\subsection{SMAWK}
	\lstinputlisting[firstline=2]{9_Else/SMAWK.cpp}
	\subsection{Python}
	\lstinputlisting[firstline=1]{11_Python/misc.py}
	\subsection{LineContainer}
	\lstinputlisting[firstline=2]{9_Else/LineContainer.cpp}


\end{document}
>>>>>>> 5c25d7a (init)
